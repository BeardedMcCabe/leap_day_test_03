%%%%%%%%%%%%%%%%%%%%
%                  %
%    Introduction  %
%                  %
%%%%%%%%%%%%%%%%%%%%

\labChapter{ii}{Introduction}



All scientists and engineers, whatever their field, use fairly standard procedures to record and analyze experimental data.
One key objective of this laboratory is that students explore the experiments using scientific methods that are common to all experiments, both in this laboratory and beyond.

The clear and correct presentation of experimental data is one of the highest priorities in science and engineering ethics.
Any claims made about experimental results must be supported through accurate measurements with related uncertainties and effective analysis.

Several concepts are important in obtaining and conveying scientific data.
\begin{enumerate}
  \item one must be able to accurately measure the variables of interest (for example, length, weight, or time) and know the limitations of those measurements.
  \item one must manipulate and convey that data in a manner that reflects the accuracy and precision of the measurements.
  \item one must be able to present the data in a format that is easily interpreted by the rest of the scientific community. Correct error analysis is critical in allowing the experimenter and others to draw valid conclusions from the results.
\end{enumerate}

In the sections at the end of this lab manual, we review the importance and use of significant figures, approaches to error analysis and the graphical representation of experimental data as well as an overview of how to use Vernier calipers for sub-millimeter length measurements.
A thorough review of this material (at the end of the manual) will greatly aid your ability to properly collect analyze and present your data in a post-laboratory submission.


