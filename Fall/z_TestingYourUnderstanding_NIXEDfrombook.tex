%experiment 1 vector table

\section{Testing your Understanding}

\begin{enumerate}
  %1
\item Determine the $x$ and $y$ component of the vector
  \begin{align*} %
    \vec{F} = 20.00\,\newton \ @ \ 60\degree
  \end{align*}
  \begin{enumerate}
  \item $F_{x} =  10.00\,\newton$ and $F_{y} = +17.32\,\newton$
  \item $F_{x} =  10.00\,\newton$ and $F_{y} = -17.32\,\newton$
  \item $F_{x} = -10.00\,\newton$ and $F_{y} = +17.32\,\newton$
  \item $F_{x} = -10.00\,\newton$ and $F_{y} = -17.32\,\newton$
  \end{enumerate}
  %2
\item Determine the $x$ and $y$ component of the vector
  \begin{align*} %
    \vec{F} = 20.00\,\newton \ @ \ 120\degree
  \end{align*}
  \begin{enumerate}
  \item $F_{x} =  10.00\,\newton$ and $F_{y} = +17.32\,\newton$
  \item $F_{x} =  10.00\,\newton$ and $F_{y} = -17.32\,\newton$
  \item $F_{x} = -10.00\,\newton$ and $F_{y} = +17.32\,\newton$
  \item $F_{x} = -10.00\,\newton$ and $F_{y} = -17.32\,\newton$
  \end{enumerate}
  %3
\item Consider the two force vectors
  \begin{equation*}
    \begin{aligned} %
      \vec{F}_1 & = 20.00\,\newton \ @ \ 0\degree \\
      \vec{F}_2 & = 20.00\,\newton \ @ \ 90\degree
    \end{aligned}
  \end{equation*}
  The resultant $\vec{R} = \vec{F}_1 + \vec{F}_2$ is
  \begin{enumerate}
  \item $\vec{R} = 20.00\,\newton \ @ \ 45\degree$
  \item $\vec{R} = 20.00\,\newton \ @ \ 225\degree$
  \item $\vec{R} = 28.28\,\newton \ @ \ 45\degree$
  \item $\vec{R} = 28.28\,\newton \ @ \ 225\degree$
  \end{enumerate}
  %4
\item Consider the two force vectors
  \begin{equation*}
    \begin{aligned} %
      \vec{F}_1 & = 20.00\,\newton \ @ \ 0\degree \\
      \vec{F}_2 & = 20.00\,\newton \ @ \ 90\degree
    \end{aligned}
  \end{equation*}
  The force $\vec{F_{3}}$ needed to fulfill the equation $\vec{F}_1 + \vec{F}_2 + \vec{F_{3}} = 0\,\newton$ is
  \begin{enumerate}
  \item $\vec{R} = 20.00\,\newton \ @ \ 45\degree$
  \item $\vec{R} = 20.00\,\newton \ @ \ 225\degree$
  \item $\vec{R} = 28.28\,\newton \ @ \ 45\degree$
  \item $\vec{R} = 28.28\,\newton \ @ \ 225\degree$
  \end{enumerate}
  %5
\item Consider the two force vectors
  \begin{equation*}
    \begin{aligned} %
      \vec{F}_1 & = 28.28\,\newton \ @ \ 45\degree \\
      \vec{F}_2 & = 20.00\,\newton\ @ \ 120\degree
    \end{aligned}
  \end{equation*}
  The resultant $\vec{R} = \vec{F}_1 + \vec{F}_2$ is represented by
  \begin{enumerate}
  \item $R_{x} = +10\,\newton$ and $R_{y} = +37.32\,\newton$
  \item $R_{x} = -10\,\newton$ and $R_{y} = -37.32\,\newton$
  \item $R_{x} = +10\,\newton$ and $R_{y} = +2.68\,\newton$
  \item $R_{x} = +10\,\newton$ and $R_{y} = -2.68\,\newton$
  \end{enumerate}
  %6
\item Consider the two force vectors
  \begin{equation*}
    \begin{aligned} %
      \vec{F}_1 & = 28.28\,\newton \ @ \ 45\degree \\
      \vec{F}_2 & = 20.00\,\newton \ @ \ 120\degree
    \end{aligned}
  \end{equation*}
  The force $\vec{F_{3}}$ needed to fulfill the equation $\vec{F}_1 + \vec{F}_2 + \vec{F_{3}} = 0\,\newton$ is
  \begin{enumerate}
  \item $\vec{F_{3}} = 38.64\,\newton\ @ \ 75\degree$
  \item $\vec{F_{3}} = 38.64\,\newton\ @ \ 225\degree$
  \item $\vec{F_{3}} = 10.35\,\newton\ @ \ 75\degree$
  \item $\vec{F_{3}} = 10.35\,\newton\ @ \ 225\degree$
  \end{enumerate}	
  %6
\item When static equilibrium is reached:
  \begin{enumerate}
  \item the ring is moving with constant, positive velocity.
  \item the ring is moving with constant, negative velocity.
  \item the ring is not moving and is located off-center, nearest the third mass.
  \item the ring is not moving and is located at the center of the force table.
  \end{enumerate}
\end{enumerate}












%experiment 2 acceleration due to gravity from track

\section{Testing your Understanding}

\begin{enumerate}
  %1
\item Assume the following values: $S = 2.20\,\meter$,  $D = 2.00\,\meter$, $H = 22.0\,\milli\meter$, and $t = 6.321\,\second$.  Determine the value of $g$ you would get for these values.
  \begin{enumerate}
  \item $g = 9.70\,\meter\per\second\squared$
  \item $g = 9.76\,\meter\per\second\squared$
  \item $g = 9.81\,\meter\per\second\squared$
  \item $g = 10.0\,\meter\per\second\squared$
  \end{enumerate}
  %2
\item  Assume the following values: $S = 2.20\,\meter$,  $D = 2.00\,\meter$, $H = 22\,\milli\meter$.  Determine the time you would need to measure to get a value of exactly $9.803\,\meter\per\second\squared$.
  \begin{enumerate}
  \item $t = 6.321\,\second$
  \item $t = 6.379\,\second$
  \item $t = 6.386\,\second$
  \item $t = 40.69\,\second$
  \end{enumerate}
  %3
\item Assume the following values: $S = 2.20\,\meter$,  $D = 2.00\,\meter$, $H = 40.0\,\milli\meter$, and $t = 4.747\,\second$.  Determine the value of $g$ you would get for these values.
  \begin{enumerate}
  \item $g = 9.70\,\meter\per\second\squared$
  \item $g = 9.76\,\meter\per\second\squared$
  \item $g = 9.81\,\meter\per\second\squared$
  \item $g = 10.0\,\meter\per\second\squared$
  \end{enumerate}
  %4
\item  Assume the following values: $S = 2.20\,\meter$,  $D = 2.00\,\meter$, $H = 40.0\,\milli\meter$.  Determine the time you would need to measure to get a value of exactly $g = 9.803\,\meter\per\second\squared$.
  \begin{enumerate}
  \item $t = 4.747\,\second$
  \item $t = 4.731\,\second$
  \item $t = 4.736\,\second$
  \item $t = 22.38\,\second$
  \end{enumerate}
  %5
\item  In the measurement of $g$, when determining the starting position of the glider you must:
  \begin{enumerate}
  \item use the same location for all cases, and this location must be directly behind the photogate.
  \item use the same location for all cases, and this location must be underneath the photogate.
  \item use different locations for all cases, and these locations must all be behind the photogate.
  \item use different locations for all cases, and these locations must be between the two photogates.
  \end{enumerate}
\end{enumerate}









%experiment 4 centripetal force

\section{Testing your Understanding}

\begin{enumerate}
  %1
\item A mass $m$ rotates at constant speed $v$ at a radius of $R$ from the rotational axis. A force $F$ provides the centripetal force $F_{c}$. If you double the mass $m$, without changing $R$ or $v$, the force needed to keep the mass in circular motion would
  \begin{enumerate}
  \item stay the same.
  \item double.
  \item halve.
  \item quadruple.
  \end{enumerate}
  %2
\item A mass $m$ rotates at constant speed $v$ at a radius of $R$ from the rotational axis. A force $F$ provides the centripetal force $F_{c}$. If you double the radius $R$ of the motion, without changing $m$ or $v$, the force needed to keep the mass in circular motion would
  \begin{enumerate}
  \item stay the same.
  \item double.
  \item halve.
  \item quadruple.
  \end{enumerate}
  %3
\item A mass $m$ rotates at constant speed $v$ at a radius of $R$ from the rotational axis. A force $F$ provides the centripetal force $F_{c}$. If you double the speed $v$ of the mass, without changing $m$ or $R$, the force needed to keep the mass in circular motion would
  \begin{enumerate}
  \item stay the same.
  \item double.
  \item halve.
  \item quadruple.
  \end{enumerate}
  %4
\item A mass $m$ rotates at constant speed $v$ at a radius of $R$ from the rotational axis. A force $F$ provides the centripetal force $F_{c}$. If you double both, the mass $m$ and the radius $R$ of the motion, without changing $v$, the force needed to keep the mass in circular motion would
  \begin{enumerate}
  \item stay the same.
  \item double.
  \item halve.
  \item quadruple.
  \end{enumerate}
  %5
\item A mass $m$ rotates at constant speed $v$ at a radius of $R$ from the rotational axis. A force $F$ provides the centripetal force $F_{c}$. If you double both, the speed $v$ of the mass and the radius $R$ of the motion, without changing $m$, the force needed to keep the mass in circular motion would
  \begin{enumerate}
  \item stay the same.
  \item double.
  \item halve.
  \item quadruple.
  \end{enumerate}
  %6
\item A mass $m$ rotates at constant speed $v$ at a radius of $R$ from the rotational axis. A force $F$ provides the centripetal force $F_{c}$. If you double $m$, $R$, and $v$, the force needed to keep the mass in circular motion would
  \begin{enumerate}
  \item stay the same.
  \item double.
  \item halve.
  \item quadruple.
  \end{enumerate}
  %7
\item When you change the mass or the radius of the free mass, you make the same change to the fixed mass because
  \begin{enumerate}
  \item the fixed mass and free mass contribute equally to the centripetal force.
  \item the fixed mass contributes less than the free mass to the centripetal force, but it is still important.
  \item the noise in the measurements will be larger due to vibration if the fixed and free masses are not balanced.
  \item the device will not rotate if the fixed and free masses are not balanced.
  \end{enumerate}
\end{enumerate}









%experiment 5 energy conservation

\section{Testing your Understanding}

\begin{enumerate}
  %1
\item Calculate the potential energy of an object with mass $m = 140\,\gram$ at a height $h = 11\,\centi\metre$. Use $g = 9.803\,\metre\per\second\squared$ for the acceleration due to gravity and assume that $U = 0\,\joule$ at $h = 0\,\centi\metre$.
  \begin{enumerate}
  \item $U = 0.151\,\joule$
  \item $U =  15.1\,\joule$
  \item $U =   151\,\joule$
  \item $U = 15100\,\joule$
  \end{enumerate}
  %2
\item Suppose an object at a given location has potential energy of $U_i = 10\,\joule$ and kinetic energy of $K_i = 20\,\joule$. Assuming there are no other outside forces acting on the object, how much kinetic energy does the object have when its potential energy is $U_f = 25\,\joule$.
  \begin{enumerate}
  \item $K_f =  5\,\joule$
  \item $K_f = 20\,\joule$
  \item $K_f = 35\,\joule$
  \item This situation cannot be achieved.
  \end{enumerate}
  %3
\item Suppose an object at a given location has potential energy of $U_i = 10\,\joule$ and kinetic energy of $K_i = 20\,\joule$. Assuming there are no other outside forces acting on the object, how much kinetic energy does the object have when its potential energy is $U_f = 40\,\joule$.
  \begin{enumerate}
  \item $K_f = -10\,\joule$
  \item $K_f =  20\,\joule$
  \item $K_f =  50\,\joule$
  \item This situation cannot be achieved.
  \end{enumerate}
  %4
\item Suppose an object at a given location has potential energy of $U_i = 10\,\joule$ and kinetic energy of $K_i = 20\,\joule$. Assuming there are no other outside forces acting on the object, how much potential energy does the object have when its kinetic energy is $K_f = 15\,\joule$.
  \begin{enumerate}
  \item $U_f =  5\,\joule$
  \item $U_f = 10\,\joule$
  \item $U_f = 15\,\joule$
  \item This situation cannot be achieved.
  \end{enumerate}
  %5
\item Suppose an object at a given location has potential energy of $U_i = 10\,\joule$ and kinetic energy of $K_i = 20\,\joule$. Assuming there are no other outside forces acting on the object, how much potential energy does the object have when its kinetic energy is $K_f = 40\,\joule$.
  \begin{enumerate}
  \item $U_f = -10\,\joule$
  \item $U_f =  10\,\joule$
  \item $U_f =  40\,\joule$
  \item This situation cannot be achieved.
  \end{enumerate}
  %6
\item Suppose an object of mass $m = 4\,\kilo\gram$ is at rest at a given location with potential energy of $U_i = 50\,\joule$. Assuming there are no other outside forces acting on the object, how fast does the object move when it reaches potential energy of $U_f = 0\,\joule$?
  \begin{enumerate}
  \item $v_f =  0\,\meter\per\second$
  \item $v_f =  5\,\meter\per\second$
  \item $v_f = 10\,\meter\per\second$
  \item $v_f = 25\,\meter\per\second$
  \end{enumerate}
  %7
\item You try to release the glider with zero initial velocity just at the point where the timer triggers because
  \begin{enumerate}
  \item if you start it further to the right it travels farther and so takes longer to reach the bottom.
  \item if you start it further to the right it has an initial velocity when it passes the timer.
  \item it doesn't really matter where you start the glider because the distance between the gates is the same.
  \end{enumerate}
\end{enumerate}






%experiment 6 Ballistic Pendulum and projectile motion

\section{Testing your Understanding}

\begin{enumerate}
  %1
\item A projectile of mass $m = 0.2\,\kilo\gram$ and initial speed $v_i = 200\,\meter\per\second$ hits a stationary target of mass $M = 2\,\kilo\gram$ and gets imbedded in it. Immediately after the collision the projectile-target combination has a speed $V$ of
  \begin{enumerate}
  \item $V = 18.2\,\meter\per\second$
  \item $V = 20.0\,\meter\per\second$
  \item $V = 60.3\,\meter\per\second$
  \item $V = 63.2\,\meter\per\second$
  \end{enumerate}
  %2
\item A projectile of mass $m = 0.2\,\kilo\gram$ and initial speed $v_i =200\,\meter\per\second$ hits a stationary target of mass $M = 2\,\kilo\gram$ and gets imbedded in it. The energy lost ($\Delta E$) during this collision is closest to
  \begin{enumerate}
  \item $\Delta E =     0\,\joule$ (no energy is lost)
  \item $\Delta E = -3640\,\joule$
  \item $\Delta E = -3670\,\joule$
  \item $\Delta E = -4000\,\joule$
  \end{enumerate}
  %3
\item A projectile of mass $m = 0.2\,\kilogram$ and initial speed $v_i = 200\,\meter\per\second$ hits a stationary target of mass $M = 2\,\kilogram$ and gets imbedded in it. Assuming that all the kinetic energy after the collision is converted into gravitational potential energy, to what maximum height above the original position will the two masses rise?
  \begin{enumerate}
  \item $h = 16.9\,\metre$
  \item $h = 18.6\,\metre$
  \item $h = 20.4\,\metre$
  \item $h = 22.5\,\metre$
  \end{enumerate}
  %4
\item A cannon fires a projectile perfectly horizontal with initial speed of $v_i = 20\,\meter\per\second$. The cannon is at an initial height of $H = 5\,\meter$. How long is the projectile in the air? For simplicity use $g = 10\,\metre\per\second\squared$.
  \begin{enumerate}
  \item $\Delta t = 1.0\,\second$
  \item $\Delta t = 1.4\,\second$
  \item $\Delta t = 2.0\,\second$
  \item $\Delta t = 4.0\,\second$
  \end{enumerate}
  %5
\item A cannon fires a projectile perfectly horizontal with initial speed of $v_i = 20\,\meter\per\second$. The cannon is at an initial height of $H = 5\,\meter$. At what horizontal distance $R$ from the launch position will the projectile hit the ground? For simplicity use $g = 10\,\meter\per\second\squared$.
  \begin{enumerate}
  \item $R = 20.0\,\meter$
  \item $R = 28.0\,\meter$
  \item $R = 40.0\,\meter$
  \item $R = 80.0\,\meter$
  \end{enumerate}
  %6
\item Suppose you have a cannon that is able to fire a projectile from an initial height $H$ with speed $v_i$. If you launch the projectile at a very slight upward angle, the range $R$ (the horizontal distance the projectile travels) will be
  \begin{enumerate}
  \item Larger as compared to if you launch the projectile horizontally.
  \item Less as compared to if you launch the projectile horizontally.
  \item The same as compared to if you launch the projectile horizontally.
  \item The answer depends on the value of the initial speed $v_i$.
  \end{enumerate}
  %7.
\item If you made only $n = 1$ measurement of rack height, the standard deviation of the measurement would be
  \begin{enumerate}
  \item zero, because there is no difference between the value and the average.
  \item Equal to the measured value.
  \item Undetermined, because you need more measurements to find a standard deviation.
  \item Infinite because the denominator $(n-1)$ is zero.
  \end{enumerate}
\end{enumerate}










%experiment 7 for 1171 Rotational Motion

\section{Testing your Understanding}

\begin{enumerate}
  %1
\item \label{item:M09Q1} Determine the moment of inertia $I$ of the Earth on its path around the Sun. Use the following values:
  \begin{align*} %
    & \text{Mass of the Earth: } M_{\mbox{E}} = 6 \times 10^{24}\,\kilo\gram\\
    & \text{Radius of the Earth's orbit around the Sun: } R_{\mbox{ES}} = 150 \times 10^{6}\,\kilo\gram
  \end{align*}
  \begin{enumerate}
  \item $I = 1.35 \times 10^{41}\,\kilo\gram \usk \meter\squared$
  \item $I = 5.40 \times 10^{46}\,\kilo\gram \usk \meter\squared$
  \item $I = 6.75 \times 10^{46}\,\kilo\gram \usk \meter\squared$
  \item $I = 1.35 \times 10^{47}\,\kilo\gram \usk \meter\squared$
  \end{enumerate}
  %2
\item Consider problem~\ref{item:M09Q1} above. How would the value you found for $I$ change if the Earth's year were only half as long (i.e.\ 182.5 days instead of 365 days)?
  \begin{enumerate}
  \item The value of $I$ would double.
  \item The value of $I$ would be half.
  \item The value of $I$ would not change.
  \item The value of $I$ would quadruple.
  \item The value of $I$ would be one quarter.
  \end{enumerate}
  %3
\item Consider problem~\ref{item:M09Q1} above. How would the value you found for $I$ change if the radius of the Earth's path around the Sun would be only half as large (i.e.\ $75 \times 10^{6}\,\kilo\meter$ instead of $150 \times 10^{6}\,\kilo\meter$)?
  \begin{enumerate}
  \item The value of $I$ would double.
  \item The value of $I$ would be half.
  \item The value of $I$ would not change.
  \item The value of $I$ would quadruple.
  \item The value of $I$ would be one quarter.
  \end{enumerate}
  %4
\item \label{M09Q4} Consider the sketch of the setup for this experiment above. Assume that the tension in the string running over the pulley is $T = 10\,\newton$ and the radius of the pulley is $R = 10\,\centi\meter$. The torque acting on the pulley is
  \begin{enumerate}
  \item   100\,\newton\metre
  \item    10\,\newton\metre
  \item     1\,\newton\metre
  \item   0.1\,\newton\metre
  \item  0.01\,\newton\metre
  \end{enumerate}
  %5
\item Consider problem~\ref{M09Q4} above. Assuming the pulley is a cylindrical, solid disk of mass $M = 5\,\kilo\gram$, the angular acceleration of the pulley is closest to
  \begin{enumerate}
  \item 0.2\,\radian\per\second\squared
  \item 2  \,\radian\per\second\squared
  \item 4  \,\radian\per\second\squared
  \item 20 \,\radian\per\second\squared
  \item 40 \,\radian\per\second\squared
  \end{enumerate}
  %6
\item During a pirouette a figure skater reduces her moment of inertia from $I_1 = 1.24\,\kilo\gram\metre\squared$ to $I_2 = 0.25\,\kilo\gram\metre\squared$ by pulling in her arms. If she rotates with an angular speed of $\omega_1 = 0.5\,\radian\per\second$, before pulling in her arms, what is her final angular speed when she finished the process?
  \begin{enumerate}
  \item  2.48\,\radian\per\second
  \item  0.10\,\radian\per\second
  \item  0.40\,\radian\per\second
  \item  9.92\,\radian\per\second
  \item  0.50\,\radian\per\second
  \end{enumerate}
  %7
\item A solid disk and a thin-walled ring of the same mass and radius roll down an incline, starting from rest and from the same height. Which of these objects will reach the bottom of the incline with a higher speed?
  \begin{enumerate}
  \item The ring.
  \item The disk.
  \item Both will reach the bottom with the same speed.
  \item More information is needed to answer this question.
  \end{enumerate}
  Hint: Think about what physics quantity is responsible for the rotation of the object and remember that both objects have the same mass and radius.
\item When measuring the smaller pulley, the reading on the calipers is:
  \begin{enumerate}
  \item The inside radius of the pulley (where the string is wound).
  \item The outside radius of the pulley.
  \item The inside diameter of the pulley.
  \item The outside diameter of the pulley.
  \end{enumerate}
\end{enumerate}







%experiment 7 for 1145 Ideal Gas Law

\section{Testing your Understanding}

\begin{enumerate}
  %1
\item Show that both sides of the Ideal Gas Law equation have units of energy (Joules).
  %2
\item Consider a fixed amount of an ideal gas undergoing an isothermal process (``iso'' is the greek word for equal and used as a prefix means that the word following it will remain the same, so here ``isothermal'' means ``constant temperature''). Under these conditions doubling the pressure $P$ of the gas will
  \begin{enumerate}
  \item Double the volume
  \item Halve the volume
  \item Not change the volume
  \item More information is needed to answer this question
  \end{enumerate}
  Note: An ideal gas under these conditions will follow Boyle's Law (sometimes also called Mariotte's Law).
  %3
\item Consider a fixed amount of an ideal gas undergoing an isobaric process (``isobaric'' means ``constant pressure''). Under these conditions doubling the temperature $T$ of the gas will
  \begin{enumerate}
  \item Double the volume
  \item Halve the volume
  \item Not change the volume
  \item More information is needed to answer this question
  \end{enumerate}
  Note: An ideal gas under these conditions will follow Gay-Lussac's Law (sometimes also called Charles' Law).
  %4
\item Consider a fixed amount of an ideal gas undergoing an isochoric process (``isochoric'' means ``constant volume''). Under these conditions doubling the temperature T of the gas will
  \begin{enumerate}
  \item Double the pressure
  \item Halve the pressure
  \item Not change the pressure
  \item More information is needed to answer this question
  \end{enumerate}
  %5
\item In the Analysis at Constant Temperature procedure, why do you have to wait for the temperature and pressure to equalize?
  \begin{enumerate}
  \item If the temperature and pressure are changing, the ideal gas law is not obeyed.
  \item It takes time for the system to come to thermal equilibrium.
  \item You must wait for the air to leak out of the apparatus, so the pressure equalizes.
  \end{enumerate}
\end{enumerate}










%experiment 9 Simple Harmonic Motion

\section{Testing your Understanding}

\begin{enumerate}
  %1
\item A mass $m = 0.4\,\kilo\gram$ is attached to a vertical hanging spring, which obeys Hooke's Law. After hanging the weight the spring is stretched from its equilibrium position by $\Delta y_0 = 0.1\,\metre$. The spring constant $k$ of the spring is
  \begin{enumerate}
  \item $k = 0.025\,\newton\per\metre$
  \item $k = 0.25 \,\newton\per\metre$
  \item $k = 4    \,\newton\per\metre$
  \item $k = 40   \,\newton\per\metre$
  \end{enumerate}
  %2
\item A mass $m = 0.4\,\kilo\gram$ is attached to a vertical hanging spring, which obeys Hooke's Law. After hanging the weight the spring is stretched from its equilibrium position by $\Delta y_0 = 0.1\,\meter$. The mass is then extended an additional $\Delta y = 0.1\,\meter$ \underline{downward} and released from rest. The period of this system is
  \begin{enumerate}
  \item $T = 0.010\,\second$
  \item $T = 0.063\,\second$
  \item $T = 0.100\,\second$
  \item $T = 0.628\,\second$
  \end{enumerate}
  %3
\item A mass $m = 0.4\,\kilo\gram$ is attached to a vertical hanging spring, which obeys Hooke's Law. After hanging the weight the spring is stretched from its equilibrium position by $\Delta y_0 = 0.1\,\meter$. The mass is then extended an additional $\Delta y = 0.1\,\meter$ \underline{upward} and released from rest. The period of this system is
  \begin{enumerate}
  \item $T = 0.010\,\second$
  \item $T = 0.063\,\second$
  \item $T = 0.100\,\second$
  \item $T = 0.628\,\second$
  \end{enumerate}
  %4
\item A mass $m = 0.4\,\kilo\gram$ is attached to a vertical hanging spring, which obeys Hooke's Law. After hanging the weight the spring is stretched from its equilibrium position by $\Delta y_0 = 0.1\,\meter$. The mass is then extended an additional $\Delta y = 0.2\,\meter$ \underline{downward} and released from rest. The period of this system is
  \begin{enumerate}
  \item $T = 0.010\,\second$
  \item $T = 0.063\,\second$
  \item $T = 0.100\,\second$
  \item $T = 0.628\,\second$
  \end{enumerate}
  %5
\item A mass $m = 0.4\,\kilo\gram$ is attached to a vertical hanging spring, which obeys Hooke's Law. After hanging the weight the spring is stretched from its equilibrium position by $\Delta y_0 = 0.1\,\meter$. The mass is then extended an additional $\Delta y = 0.2\,\meter$ \underline{upward} and released from rest. The period of this system is
  \begin{enumerate}
  \item $T = 0.010\,\second$
  \item $T = 0.063\,\second$
  \item $T = 0.100\,\second$
  \item $T = 0.628\,\second$
  \end{enumerate}
  %6
\item A mass $m = 0.05\,\kilo\gram$ is attached to a vertical hanging spring, which obeys Hooke's Law ($k = 100\,\newton\per\metre$). The mass is then extended an additional $\Delta y = 0.1\,\meter$ \underline{downward} and released from rest. After 10 oscillations the amplitude is observered to be $\Delta y = 0.02\,\meter$. The energy lost in the system is
  \begin{enumerate}
  \item $\Delta E = 0.18\,\joule$
  \item $\Delta E = 0.48\,\joule$
  \item $\Delta E = 1.0\,\joule$
  \item $\Delta E = 4.0\,\joule$
  \end{enumerate}
  %6
\item When measuring the period of the glider with the sail, why do you stop the glider after 5 cycles?
  \begin{enumerate}
  \item That is enough to measure an accurate period.
  \item The motion of the glider becomes too small after 5 cycles to accurately determine when to stop the stopwatch.
  \item Five cycles is a compromise with enough cycles to accurately measure the period while the amplitude is roughly the same.
  \item The sail increases the period of the glider so much that it would take too long to measure 10 cycles.
  \end{enumerate}
  %7
\item Which part of the glider should you use when determining the position on the track?
  \begin{enumerate}
  \item The front corner.
  \item The back corner.
  \item Consistently either the front or the back corner.
  \item The center of mass.
  \end{enumerate}

\end{enumerate}





%experiment 10 Simple Pendulum Solve for G

\section{Testing your Understanding}

\begin{enumerate}
  %1
\item Using the small-angle approximation, determine the period $T$ for a pendulum of $L = 1.5\,\meter$ and angle $\theta = 3\degree$. Use $g = 9.803\,\meter\per\second\squared$ for the acceleration due to gravity.
  \begin{enumerate}
  \item $T = 0.39\,\second$
  \item $T = 0.96\,\second$
  \item $T = 1.23\,\second$
  \item $T = 2.45\,\second$
  \end{enumerate}
  %2
\item Using the complete solution (equation~\ref{eq:M03PeriodSeries}), determine the period $T$ for a pendulum of $L = 1.5\,\meter$ and angle $\theta = 3\degree$. Use $g = 9.803\,\meter\per\second\squared$ for the acceleration due to gravity.
  \begin{enumerate}
  \item $T =  0.39\,\second$
  \item $T =  0.96\,\second$
  \item $T =  1.23\,\second$
  \item $T =  2.45\,\second$
  \end{enumerate}
  %3
\item Using the small-angle approximation, determine the period $T$ for a pendulum of $L = 1.5\,\meter$ and angle $\theta = 45\degree$. Use $g = 9.803\,\meter\per\second\squared$ for the acceleration due to gravity.
  \begin{enumerate}
  \item $T =  0.39\,\second$
  \item $T =  0.96\,\second$
  \item $T =  1.23\,\second$
  \item $T =  2.45\,\second$
  \end{enumerate}
  %4
\item Using the complete solution (equation~\ref{eq:M03PeriodSeries}), determine the period $T$ for a pendulum of $L = 1.5\,\meter$ and angle $\theta = 45\degree$. Use $g = 9.803\,\meter\per\second\squared$ for the acceleration due to gravity.
  \begin{enumerate}
  \item $T =  0.41\,\second$
  \item $T =  1.00\,\second$
  \item $T =  1.28\,\second$
  \item $T =  2.55\,\second$
  \end{enumerate}
  % 5
\item If you were to perform this experiment in Denver, CO instead of Fairfield U, how would you expect the period $T$ to change? Assume the exact, identical experimental conditions
  \begin{enumerate}
  \item The period $T$ would stay the same.
  \item The period $T$ would increase.
  \item The period $T$ would decrease.
  \end{enumerate}
  % 6
\item When setting the length of the pendulum, it is important that:
  \begin{enumerate}
  \item The length of string from the pivot to the surface of the ball is exactly the stated length for the case.
  \item The length of string from the pivot to the center of the ball is exactly the stated length for the case.
  \item The accurately measure length of the string plus the radius of the ball is used in the calculations.
  \end{enumerate}
\end{enumerate}


