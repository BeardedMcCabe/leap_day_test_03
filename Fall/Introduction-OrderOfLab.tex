%%%%%%%%%%%%%%%%%%%%
%                  %
%    Order of Lab  %
%                  %
%%%%%%%%%%%%%%%%%%%%
\labChapter{ii}{Order Of A Lab}


%\textbf{The general order of the labs will be:}
\textbf{READ \& REVIEW THE LAB MANUAL \textit{BEFORE} COMING TO LAB}

%\textbf{!!!!!!!!!!!!!!!!!!!!!!!!!!!!!!!!!!!!!!!!!!!!!!!!!!!!!!!!!!!!!!!!!!! !!!!!!!!!!!!!!!!!!!!!!!!!!!!!!!!!!!!!!!!!!!!!!!! Preliminary percentages !!!!!!!!!!!!!!!!!!!!!!!!!!!!!!!!!!!!!!!!! !!!!!!!!!!!!!!!!!!!!!!!!!!!!!!!!!!!!!!!!!!!!!!}
%!!!!!!  \textbf{DRAFT VERSION HERE}  !!!!!!


\begin{itemize}
    \item \textbf{Pre-lab homework [10\%]}
    \begin{itemize}
        \item \textbf{INDIVIDUAL}
        \item \textbf{OPEN BOOK}
        \item Through Blackboard. Due \textbf{\textit{before}} class.
        \item Intended to give you a chance to review the physical concepts and equations.
    \end{itemize}
    \item \textbf{Beginning-of-lab quiz [15\%]}
        \begin{itemize}
            \item \textbf{INDIVIDUAL}
            \item \textbf{CLOSED BOOK}
            \item To ensure you are prepared for the lab, they will incorporate discussions like:
            \begin{itemize}
                \item What is the goal of the experiment?
                \item What quantity(ies) do you want to determine and/or compare?
                \item What quantity(ies) will you measure to determine your result?
                \item What Physics principle(s) and/or assumptions will you use?
            \end{itemize}
        \end{itemize}
    \item \textbf{In-lab Spreadsheet [10\%, \textit{Pass/Fail}]}
        \begin{itemize}
            \item  \textbf{INDIVIDUAL for labs E-1 -- E-4}
            \item  \textbf{GROUP for labs E-5 -- W-10}
            \item Either submit to Blackboard before leaving or get checked out by professor; \textbf{\textit{check syllabus for your specific course}}.
            \item Spreadsheet made in class by each person individually (same data as your group of course). This will switch to where everyone can work collaboratively after you've practiced individually for the first four labs.
      %    \textbf{  \item !!!!!!!!!!!!!!!!!!!!!! Something about sharing their spreadsheets or data between group mates -- like some end up using Google Sheets or Excel online and make a collaborative spreadsheet (usually leading to one person doing most of the work). Option to deal with that is that if they want to share spreadsheets after lab, they can only share a PDF (Bob suggestion). Otherwise if people have the exact same spreadsheet from a group --- lots of points off? Should we have an easier way to deal with Pass/Fail --- like signing out, or do we want to give feedback on the initial draft? Build into the Blackboard gradebook as pass/fail or like 0/5/10 points (none, good but with feedback, good)?!!!!!!!!!}
            \item \textbf{MUST BE DONE IN CLASS}
            \item Can be rough, but \textbf{must show}:
            \begin{itemize}
                \item Completed experiment with \textbf{data}, \textbf{uncertainties}, \textbf{rough graphs}
                \item Example of how errors affect final results (e.g. result $\pm$ value or range)
                \item All calculations completed in Excel (not just copied from calculator/paper)
                \item For calculations, reference your common data (e.g. $=\$A\$2$, dollar signs used to hold columns and/or rows static)
            \end{itemize}
        \end{itemize}
    \item \textbf{Post-lab submission [65\%] - INDIVIDUAL Spreadsheet \& Interpretation}

    \begin{itemize}
        \item Finalized Spreadsheet \textbf{[25\%]}
        \begin{itemize}
            \item  \textbf{INDIVIDUAL, clean up your own spreadsheet}
            \item Organized and cleaned-up data with reasonable significant figures, summaries of overall results with their averages, uncertainties, standard deviations, and formatted plots.
            \item Again, all calculations must be present in spreadsheet. \textbf{Don't just copy values from paper or calculator.}
        \end{itemize}
        \item Post-lab Interpretation \textbf{[40\%]}
        \begin{itemize}
            \item \textbf{INDIVIDUAL}
            \item Paragraph discussing results +/- uncertainties, trends and relationships.
            \item Paragraph discussing errors due to systematic errors (bias, noise, friction, etc.) and statistical or random errors (e.g. measurement uncertainties) and how they affect your final results. 
            \item Discussion of what you learned.
            \item Correctly formatted in English.
        \end{itemize}
    \end{itemize}
%    \begin{itemize}
%       \item Final cleaned up spreadsheet with finished plots
%        \item Paragraph of your results +/- standard deviations from your data.
%        
%        
%        Answers to the Interpretation of Results questions for the analysis, such as:
  %      \begin{itemize}
  %  \item Describe the physical process or principle being examined in lab.
  %  \item Examine how your results relate to this physical process.
  %  \item Describe how your results provide evidence for this physical process.
  %  \item Identify trends in the data.
  %  \item Describe the sources of measurement error / uncertainties. Measurement errors can stem from the instrumentation, the experimental setup and procedure. Estimate the magnitude of measurement uncertainties. 
  %  \item Describe steps you took to minimize those measurement errors.
 %   \item Explain whether errors are systematic or random.
%    \item Compare the difference between your results and expected results with the magnitude of error you expect.

   %     \end{itemize}

        
%    \end{itemize}
\end{itemize}

%Some points that you should consider in your laboratory reports include:
%Some points that you will consider in your post-lab submissions include:
%\begin{itemize}
%    \item Describe the physical process or principle being examined in lab.
%    \item Examine how your results relate to this physical process.
%    \item Describe how your results provide evidence for this physical process.
%    \item Identify trends in the data.
%    \item Describe the sources of measurement error / uncertainties. Measurement errors can stem from the instrumentation, the experimental setup and procedure. Estimate the magnitude of measurement uncertainties. 
%    \item Describe steps you took to minimize those measurement errors.
%    \item Explain whether errors are systematic or random.
%    \item Compare the difference between your results and expected results with the magnitude of error you expect.
%\end{itemize}


